% Created 2012-10-23 Tue 10:53
\documentclass[11pt]{article}
\usepackage[utf8]{inputenc}
\usepackage[T1]{fontenc}
\usepackage{fixltx2e}
\usepackage{graphicx}
\usepackage{longtable}
\usepackage{float}
\usepackage{wrapfig}
\usepackage{soul}
\usepackage{textcomp}
\usepackage{marvosym}
\usepackage{wasysym}
\usepackage{latexsym}
\usepackage{amssymb}
\usepackage{hyperref}
\tolerance=1000
\providecommand{\alert}[1]{\textbf{#1}}

\title{chefdocumentation}
\author{Scott M. Likens}
\date{\today}
\hypersetup{
  pdfkeywords={},
  pdfsubject={},
  pdfcreator={Emacs Org-mode version 7.8.11}}

\begin{document}

\maketitle

\setcounter{tocdepth}{3}
\tableofcontents
\vspace*{1cm}
\section{New Chef Documentation}
\label{sec-1}
\subsection{Changing to Sphinx}
\label{sec-1-1}

Allows to manage documenation as code
\subsubsection{Documentation moves to Github}
\label{sec-1-1-1}

\begin{itemize}
\item chef-docs will be the repo-name, it currently is private.
  each version will have a tag, so you can build the documentation for each version.
  creative commons, no ccla required.  
  If you need to update the documentation, fork, issue a pull request and make a ticket.
\begin{itemize}
\item Weekly documentation Review
\begin{itemize}
\item Just like they do for the regular chef code
\end{itemize}
\end{itemize}
\end{itemize}

\end{document}
